	%------------------------------------------------------------------
\thispagestyle{empty}
{\setlength{\parindent}{0cm}\bf{\sf The QUESO Library}}\hfill $~$\\
\begin{picture}(8,0.1)
\linethickness{3pt}
\put(0,0.1){\line(1,0){6.6}}
\end{picture}

\begin{flushright}
\sf
User's Manual\\
Version 0.47.1\\
%October 30, 2012\\
\end{flushright}


\vfill

\begin{center}
\begin{LARGE}
\sf\bf 
Quantification of Uncertainty for Estimation,\\
Simulation, and Optimization (QUESO)\\
\end{LARGE}
\end{center}


\vfill
%$~$\\
%{\bf Lead Developer:}\hfill \\
%$~\hspace{10pt}$ {\em{Ernesto E. Prudencio}}\hfill\\ 
$~$\\
% {\bf Contributors:}\hfill \\
% $~\hspace{10pt}$ {\em{Paul T. Bauman}}  \hfill \\
% $~\hspace{10pt}$ {\em{Sai Hung Cheung}} \hfill \\
% $~\hspace{10pt}$ {\em{Kemelli C. Estacio-Hiroms}} \hfill \\
% $~\hspace{10pt}$ {\em{Todd A. Oliver}}  \hfill \\
% $~\hspace{10pt}$ {\em{Ernesto E. Prudencio}} \hfill\\ 
% $~\hspace{10pt}$ {\em{Karl W. Schulz}}  \hfill \\
% $~\hspace{10pt}$ {\em{Rhys Ulerich}}    \hfill \\

\noindent
{\bf\sf Editors:}\hfill \\
{\sf Kemelli C. Estacio-Hiroms}  \\
{\sf Ernesto E. Prudencio} \\ 


\vfill

\begin{minipage}[b]{0.20\linewidth}
\includegraphics[height=4\baselineskip]{figs/ices_logo}
\end{minipage}
\hfill
\begin{minipage}[b]{0.80\linewidth}
\small\sf
Center for Predictive Engineering and Computational Sciences (PECOS) \hfill\\
Institute for Computational and Engineering Sciences (ICES) \hfill\\
The University of Texas at Austin\hfill\\
Austin, TX 78712, USA
\end{minipage}
$~$\\
\begin{picture}(8,0.1)
\linethickness{1.5pt}
\put(0,0.1){\line(1,0){6.6}}
\end{picture}

\clearpage
%------------------------------------------------------------------
\thispagestyle{empty}
$~$\\
\vfill
Copyright \copyright\ 2008-2013 The PECOS Development Team, \texttt{http://pecos.ices.utexas.edu}\\
Permission is granted to copy, distribute and/or modify this document under the terms of
the GNU Free Documentation License, Version 1.2 or any later version published by the Free
Software Foundation; with the Invariant Sections being ``GNU General Public License'' and
``Free Software Needs Free Documentation'', the Front-Cover text being ``A GNU Manual'',
and with the Back-Cover text being ``You have the freedom to copy and modify this GNU Manual''.
A copy  of the license is included in the section entitled ``GNU Free Documentation License''.

\clearpage
%------------------------------------------------------------------
\addcontentsline{toc}{chapter}{Abstract}
%\thispagestyle{empty}
\centerline{\LARGE\sffamily Abstract}
$~$\\

QUESO stands for Quantification of Uncertainty for Estimation, Simulation and Optimization and consists of 
 a collection of algorithms and C++ classes intended for
research in uncertainty quantification,
including
the solution of statistical inverse and statistical forward problems,
the validation of mathematical models under uncertainty, and
the prediction of quantities of interest from such models along with
the quantification of their uncertainties.

QUESO is designed for flexibility, portability, easy of use and
easy of extension. Its software design follows an object-oriented
approach and its code is written on C++ and over MPI. It can run over
uniprocessor or multiprocessor environments.

QUESO contains two forms of documentation:
a user's manual available in pdf format
and
a lower-level code documentation available in web based/html format.

This is the user's manual: it gives an overview of the QUESO capabilities,
provides procedures for software execution, and includes example studies.

\clearpage
%------------------------------------------------------------------
$~$\\

\clearpage
%------------------------------------------------------------------
\addcontentsline{toc}{chapter}{Disclaimer}
%\thispagestyle{empty}
\centerline{\LARGE\sffamily Disclaimer}
$~$\\
    This document was prepared
    by The University of Texas at Austin.
    Neither the University of Texas
    at Austin, nor any of its institutes, departments and employees, make any warranty, express or implied,
    or assume any legal liability or responsibility for the accuracy, completeness, or
    usefulness of any information, apparatus, product, or process disclosed, or represent
    that its use would not infringe privately owned rights. Reference herein to any specific
    commercial product, process, or service by trade name, trademark, manufacturer, or otherwise,
    does not necessarily constitute or imply its endorsement, recommendation, or favoring by
    The University of Texas at Austin or any of its institutes, departments and employees thereof.
    The views and opinions expressed herein do not necessarily state or reflect
    those of The University of Texas at Austin or any institute or department
    thereof.

    \new{
	QUESO library as well as this material are provided as is, with absolutely no warranty
    expressed or implied.  Any use is at your own risk.}

\clearpage
%------------------------------------------------------------------
$~$\\

\clearpage
%------------------------------------------------------------------
{\markboth{}{}
\addtocontents{toc}{\protect\markboth{}{}}
}
%\addtocontents{toc}{\protect\thispagestyle{headings}}
\tableofcontents 

%\clearpage
%%------------------------------------------------------------------
%$~$\\

\clearpage
%------------------------------------------------------------------
\addcontentsline{toc}{chapter}{Preface}
\thispagestyle{empty}
%\centerline{\Large\bf Preface}
\chapter*{Preface}
$~$\\
The QUESO project started in 2008 as part
of the efforts of the recently established Center for Predictive Engineering and Computational Sciences (PECOS)
at the Institute for Computational and Engineering Sciences (ICES) at The University of Texas at Austin.

The PECOS Center was selected by the National Nuclear Security Administration (NNSA) as one of its new five centers of excellence
under the Predictive Science Academic Alliance Program (PSAAP).
The goal of the PECOS Center is
to advance predictive science and to develop the next generation of advanced computational methods and tools
for the calculation of reliable predictions on the behavior of complex phenomena and systems (multiscale, multidisciplinary).
This objective demands a systematic, comprehensive treatment of the calibration and validation of the mathematical models involved,
as well as the quantification of the uncertainties inherent in such models.
The advancement of predictive science is essential for the application of Computational Science to the solution of realistic problems of national interest.

The QUESO library, since its first version, has been publicly released as open source
under the GNU General Public License and is available for free download world-wide.
See http://www.gnu.org/licenses/gpl.html for more information on the GPL software use agreement.\\

% The QUESO development team currently consists of
% Paul T. Bauman,
% Sai Hung Cheung,
% Todd A. Oliver,
% Ernesto E. Prudencio,
% Karl W. Schulz, and
% Rhys Ulerich.\\
\noindent
{\bf Contact Information:}\\
Paul T. Bauman,
Kemelli C. Estacio-Hiroms or
Damon McDougall,
\\
Institute for Computational and Engineering Sciences\\
1 University Station C0200\\
Austin, Texas 78712\\
email: pecos-dev@ices.utexas.edu\\
web: http://pecos.ices.utexas.edu\\
$~$\\

% \centerline{\bf Referencing the QUESO Library}
\section*{Referencing the QUESO Library}
When referencing the QUESO library in a publication, please cite the following:
\begin{verbatim}
@incollection{QUESO,
  author    = "Ernesto Prudencio and Karl W. Schulz",
  title     = {The Parallel C++ Statistical Library `QUESO':
               Quantification of Uncertainty for Estimation,
               Simulation and Optimization},
  booktitle = {Euro-Par 2011: Parallel Processing Workshops},
  series    = {Lecture Notes in Computer Science},
  publisher = {Springer Berlin / Heidelberg},
  isbn      = {978-3-642-29736-6},
  keyword   = {Computer Science},
  pages     = {398-407},
  volume    = {7155},
  url       = {http://dx.doi.org/10.1007/978-3-642-29737-3_44},
  year      = {2012}
}

@TechReport{queso-user-ref,
   Author      = {Kemelli C. Estacio-Hiroms and Ernesto E. Prudencio},
   Title       = {{T}he {QUESO} {L}ibrary, {U}ser's {M}anual},
   Institution = {Center for Predictive Engineering and Computational Sciences
                  (PECOS), at the Institute for Computational and Engineering
                  Sciences (ICES), The University of Texas at Austin},
   Note        = {in preparation},
   Year        = {2013}
}


@Misc{queso-web-page,
  Author = {{QUESO} Development Team},
  Title  = {{T}he {QUESO} {L}ibrary: {Q}uantification of {U}ncertainty
            for {E}stimation, {S}imulation and {O}ptimization},
  Note   = \url{https://github.com/libqueso/},
  Year   = {2008-2013}
}

\end{verbatim}
% $~$\\
% $~$\\
% 
% 
% \centerline{\bf \Queso{} Development Team}
\section*{\Queso{} Development Team}

The QUESO development team currently consists of
Paul T. Bauman,
Sai Hung Cheung,
Kemelli C. Estacio-Hiroms,
Nicholas Malaya,
Damon McDougall,
Kenji Miki,
Todd A. Oliver,
Ernesto E. Prudencio,
Karl W. Schulz, 
Chris Simmons, and
Rhys Ulerich.

% $~$\\
% $~$\\
% 
% \centerline{\bf Acknowledgments}
\section*{Acknowledgments}

This work has been supported by the United States Department of Energy,
under the National Nuclear Security Administration Predictive Science Academic Alliance Program (PSAAP) award number [DE-FC52-08NA28615], and
under the Office of Science Scientific Discovery through Advanced Computing (SciDAC) award number [DE-SC0006656].


We would also like to thank
James Martin,
Roy Stogner and
Lucas Wilcox
for interesting discussions and constructive feedback.

%\clearpage
%%------------------------------------------------------------------
%$~$\\
% 
% $~$\\
% 
% \centerline{\bf Target Audience}
\section*{Target Audience}

% \new{
% The target audience for this manual is researchers who are comfortable with UNIX concepts and the command line, have basic knowledge of a programming language, preferably C/C++, and have solid background in Bayesian methods.%, such as Monte Carlo Methods.
% }
% %  
% % The authors have assumed that readers of this manual possess some knowledge of Bayesian methods and are comfortable with UNIX concepts and the command line.
% % %%The authors take as their target audience individuals with solid background in statistics, including some familiarity with basic Bayesian methods.

\vspace{10pt}
\new{
% Thanks, Ernesto!

QUESO is a collection of statistical algorithms and programming constructs supporting research into the uncertainty quantification (UQ) of models and their predictions. UQ may be a very complex and time consuming task, involving many steps: 
decide which physical model(s) to use;
decide which reference or experimental data to use;
decide which discrepancy models to use;
decide which quantity(ies) of interest (QoI) to compute; 
decide which parameters to calibrate;
perform computational runs and collect results;
analyze computational results, and eventually reiterate;
predict QoI(s) with uncertainty.

The purpose of this manual is \underline{not} to teach UQ and its methods, but rather to introduce QUESO library so it can be used as a tool to assist and facilitate the uncertainty quantification of the user's application.
%
Thus, the target audience of this manual is researchers who have solid background in Bayesian methods, are  comfortable with UNIX concepts and the command line, and have  knowledge of a programming language, preferably
C/C++. Bellow we suggest some useful literature:
\vspace{-4pt}
\begin{enumerate}
\item Probability, statistics, random variables \cite{DasGupta2008,Durret2005,JacodProtter2004};\vspace{-9pt}
\item Bayes' formula \cite{CarlinLouis2009,GelmanEtAl2004,Jaynes2003,Ro04};\vspace{-9pt}
\item Markov chain Monte Carlo (MCMC) methods \cite{CaSo07,GrMi01,HaLaMiSa06,HaSaTa01,Hast_1970,KaSo05,Laine08,Metr_1953,Mira01};\vspace{-9pt}
\item Monte Carlo methods \cite{RoCa04};\vspace{-9pt}
\item Kernel density estimation \cite{Silverman1986};\vspace{-9pt}
\item C++ \cite{StlJosuttis1999,CppLipman2005};\vspace{-9pt}
\item Message Passing Interface (MPI) \cite{Openmpi,Mpich};\vspace{-9pt}
\item UNIX/Linux (installation of packages, compilation, linking);\vspace{-9pt}
\item MATLAB/GNU Octave (for dealing with output files generated by QUESO); and \vspace{-9pt}
\item UQ issues in general \cite{VvUqReport2012}.\vspace{-9pt}
\end{enumerate}
}
